\documentclass[11pt]{article}
\usepackage[utf8]{inputenc} 
\usepackage{a4wide} 
\usepackage[russian]{babel} 
\usepackage{graphicx} 
\usepackage{amsmath} 
\usepackage{amsfonts} 
\usepackage{amssymb} 
\usepackage{amsthm}
 \usepackage{float}

\newtheorem{definition}{Определение}[section] 
\newtheorem{theorem}{Теорема}[section] 
\newtheorem{property}{Свойство}[section] 
\newtheorem{corollary}{Следствие}[theorem] 
\newtheorem{lemma}[theorem]{Лемма} 
\newtheorem*{statement}{Утверждение} 
\renewenvironment{proof}{
\noindent\textit{Доказательство: }} {\qed}



\begin{document}
	%\thispagestyle{empty}
\begin{center}
    \ \vspace{-3cm}
    
    \includegraphics[width=0.5
    \textwidth]{msu}\\
    {\scshape Московский государственный университет имени М.~В.~Ломоносова}\\
    Факультет вычислительной математики и кибернетики\\
    Кафедра системного анализа
    
    \vfill
    
    {\LARGE Отчёт по практикуму}
    
    \vspace{1cm}
    
    {\Huge\bfseries "<Нелинейная задача оптимального управления">\\ (задача управления тележкой)} 
\end{center}
\vspace{1cm}
\begin{flushright}
    \large \textit{Студент 315 группы}\\
    В.\,В.~Кожемяк
    
    \vspace{5mm}
    
    \textit{Руководители практикума}\\
    к.ф.-м.н., доцент П.\,А.~Точилин 
\end{flushright}

\vfill
\begin{center}
    Москва, 2018 
\end{center}
\pagebreak

%Contents
\tableofcontents

\pagebreak


\section{Постановка задачи}
	Движение материальной точки на прямой описывается обыкновенным дифференциальным уравнением:
	$$
		\ddot{x} = u_1 - \dot{x}(1 + u_2), \; t \in [0, T],
	$$
	где $ x \in \mathbb{R}, \; u = (u_1, u_2)^T \in \mathbb{R}^2. $ На возможные значения управляющих параметров $ u_1, u_2 $ наложены следующие ограничения:
	\begin{enumerate}
		\item либо $ u_1 \geqslant 0, \; u_2 \in [0, k], \; k > 0, $
		\item либо $ u_1 \in \mathbb{R}, \; u_2 \in [0, k], \; k > 0. $
	\end{enumerate}
	Задан начальный момент времени $ t_0 = 0. $ Начальная позиция $ x(0) $ может быть выбрана произвольной, но так, чтобы выполнялось условие $ x(0) \in [x_{min}, x_{max}]. $ В начальный момент времени материальная точка неподвижна: $ \dot{x}(0) = 0. $ Необходимо за счёт выбора программного управления $ u $ и начальной позиции $ x(0) $ перевести материальную точку из начальной позиции в такую позицию в момент времени $ T, $ в которой $ x(T) = L > x_{max}, \; \vert \dot{x}(T) \vert \leqslant \varepsilon. $ На множестве всех реализаций программных управлений, переводящих материальную точку в указанное множество, необходимо решить задачу оптимизации:
	$$
		J = \int \limits_0^T u_1^2(t) dt \rightarrow \min \limits_{ u(\cdot) }
	$$
	\begin{enumerate}
		\item Необходимо написать в среде \texttt{Matlab} программу с пользовательским интерфейсом, которая по заданным параметрам $ T, k, x_{min}, x_{max}, L, \varepsilon $ определяет, разрешима ли задача оптимального управления (при одном из указанных двух ограничений на управления). Если задача разрешима, то программа должна построить графики компонент оптимального управления, компонент оптимальной траектории, сопряжённых переменных. Кроме того, программа должна определить количество переключений найденного оптимального управления, а также моменты переключений.
		\item В соответствующем задания отчёте необходимо привести все теоретические выкладки, сделанные в ходе решения задачи оптимального управления, привести примеры построенных оптимальных управлений и траекторий (с иллюстрациями). Все вспомогательные утверждения (за исключением принципа максимума Понтрягина), указанные в отчёте, должны быть доказаны. В отчёте должны быть приведены примеры оптимальных траекторий для всех возможных качественно различных ``режимов''.
	\end{enumerate}
	\textbf{Замечание.} Алгоритм построения оптимальной траектории не должен содержать перебра по параметрам, значения которых не ограничены, а также по более чем двум скалярным параматрам.
	
\section{Алгоритм решения}
	Обозначим $ x = x_1, \; \dot{x} = x_2 $ и запишем обыкновенное дифференциальное уравнение второго порядка:
	$$
		\ddot{x} = u_1 - \dot{x}(1 + u_2), \; t \in [0, T],
	$$
	в виде системы обыкновенных дифференциальных уравнений первого порядка
	\begin{equation}
		\label{odeSystem}
		\begin{cases}
   			\dot{x_1} = x_2, \\ 
   			\dot{x_2} = u_1 - x_2(1 + u_2),
 		\end{cases}
	\end{equation}
	с краевыми условиями
	\begin{equation*}
		\begin{cases}
   			x_1(0) = 0, \\
   			x_2(0) = 0, \\
   			x_1(T) = L, \\
   			\vert x_2(T) \vert \leqslant \varepsilon.
 		\end{cases}
	\end{equation*}
	Рассмотрим уравнение 
	$$
		\dot{x_2} = u_1 - x_2(1 + u_2). 
	$$
	Решение данного уравнения даёт формула Коши:
	$$
		x_2(t) = \chi(t, t_0)x^0 + \int \limits_0^t \chi(t, \tau)f(\tau) d\tau,
	$$
	где $ \chi(t, \tau) = \exp \left(- \int \limits_\tau^t (1 + u_2(s)) ds \right). $ \\
	Подставляя $ f(\tau) = u_1, x^0 = (0, 0)^T $ в $ x_2(t), $ полчаем
	\begin{equation}
	\label{x2(t)}
		x_2(t) = \int \limits_0^t u_1(\tau) \cdot \exp \left(- \int \limits_\tau^t (1 + u_2(s)) ds \right) d\tau
	\end{equation}
	Далее запишем функцию Гамильтона-Понтрягина
	$$
		H = \psi_0 u_1^2 + \psi_1 x_2 + \psi_2 (u_1 - x_2(1 + u_2)), 
	$$
	и сопряжённую систему
	\begin{equation}
		\label{conjSystem}
		\begin{cases}
			\dot{\psi_1} = 0, \\
			\dot{\psi_2} = - \psi_1 + \psi_2(1 + u_2).
   		\end{cases}
	\end{equation}
	Так как $ \dot{\psi_1} = 0, $ следовательно $ \psi_1 \equiv const. $
	
\subsection{Случай, когда $ u_1 \geqslant 0, \; u_2 \in [0, k], k > 0 $}
	\subsubsection{Случай, когда $ \psi_0 \neq 0 $}
	 Из условия максимума определим оптимальное управление $ u^{\star}(t) = (u_1^{\star}(t), u_2^{\star}(t)). $ А для того чтобы избавиться от свободы выбора $ \psi^{\star}(\cdot), $ рассмотрим ``нормальный'' случай, т.е $ \psi_0 < 0. $ Без ограничения общности положим $ \psi_0 = -\dfrac{1}{2}. $
	\begin{equation*}
		u_1^{\star} = 
		\begin{cases}
			\psi_2, & \psi_2 \geqslant 0, \\
			0, & \psi_2 < 0, \\   		
		\end{cases}
   	\end{equation*}
   	\begin{equation*}
   		u_2^{\star} = 
		\begin{cases}
			0, & \psi_2 x_2 > 0, \\
			[0, k], & \psi_2 x_2 = 0, \\ 
			k, & \psi_2 x_2 < 0.
   		\end{cases}
	\end{equation*}
	Временно положим $ \psi_2(0) = \psi_2^0. $
	\begin{enumerate}
		\item Пусть $ \psi_2^0 > 0. $ \\
		Так как $ \psi_2^0 > 0 $ и $ \psi_2(t) $ --- непрерывна, то $ \exists \delta > 0: \psi_2(t) > 0, \; \forall t \in [0, \delta]. $
		Следовательно $ u_1(t) = \psi_2(t) \; \forall t \in [0, \delta]. $ Подставляя найденное $ u_1(t) $ в формулу \eqref{x2(t)}, получим
		$$
			x_2(t) = \int \limits_0^t \psi_2(\tau) \cdot \exp \left(- \int \limits_\tau^t (1 + u_2(s)) ds \right) d\tau, \; \forall t \in [0, \delta].
		$$ 
		Пользуясь тем, что интеграл от неотрицательной подинтегральной функции  неотрицателен, можно сделать вывод: $ x_2(t) \geqslant 0, \; \forall t \geqslant 0. $ \\
		Теперь, зная $ u_1(t), u_2(t) $, можно записать новую систему, состощую из вторых уравнений систем \eqref{odeSystem} и \eqref{conjSystem}
		\begin{equation*}
			\begin{cases}
				\dot{x_2} = \psi_2 - x_2, \\
				\dot{\psi_2} = -\psi_1 + \psi_2.   		
			\end{cases}
   		\end{equation*}
   		Теперь решим эту систему. Начнём со второго уравнения
   		$$
   			\dot{\psi_2} = -\psi_1 + \psi_2.
   		$$
   		Решать будем методом вариации постоянной, помня при этом, что $ \psi_1 \equiv const. $ И так, для начала решаем однородную 
систему
   		\begin{gather*}
   			\dot{\psi_2} = \psi_2 \\
   			\dfrac{d \psi_2(t)}{\psi_2(t)} = dt \\
   			\ln \vert \psi_2(t) \vert = t + \tilde{C} \\
   			\psi_2(t) = C e^t.
   		\end{gather*}
   		Далее $ C \rightarrow C(t) $ и подставляем в исходное уравнение
   		\begin{gather*}
   			C'(t) e^t + C(t) e^t = -\psi_1 + C(t) e^t \\
   			C'(t) = - \psi_1 e^{-t} \\
   			C(t) = \psi_1 e^{-t} + \tilde{C}.
   		\end{gather*}
   		Теперь подставляем $ C(t) $ в уравнение для $ \psi_2(t) $
   		\begin{gather*}
			\psi_2(t) = \psi_1 + \tilde{C} e^t.   			
   		\end{gather*}
   		Подставляя краевое условие $ \psi_2(0) = \psi_2^0, $ в результате получим
   		\begin{gather*}
			\psi_2(t) = \psi_1 + (\psi_2^0 - \psi_1) e^t.   			
   		\end{gather*}
   		Теперь, зная $ \psi_2(t), $ можно найти
   		\begin{gather*}
			\dot{x_2}(t) = \psi_2(t) - x_2(t).   			
   		\end{gather*}
   		\begin{itemize}
   			\item $ \psi_1 < \psi_2^0. $ \\ 
   			Для этого воспользуемся методом вариации постоянной. Для начала решим однородное уравнение
   			\begin{gather*}
				\dot{x_2}(t) = - x_2(t) \\
				\dfrac{d x_2(t)}{x_2(t)} = -dt \\
   				\ln \vert x_2(t) \vert = -t + \tilde{C} \\
   				x_2(t) = C e^{-t}.  			
   			\end{gather*}
   			Далее $ C \rightarrow C(t) $ и подставляем в исходное уравнение
   			\begin{gather*}
				C'(t) e^{-t} - C(t) e^{-t} = \psi_1 + (\psi_2^0 - \psi_1) e^t - C(t) e^{-t} \\
				C'(t) = \psi_1 e^t + (\psi_2^0 - \psi_1) e^{2t} \\
				C(t) = \psi_1 e^t + \frac{1}{2}(\psi_2^0 - \psi_1) e^{2t} + \tilde{C}.			  		
   			\end{gather*}
   			Теперь подставляем $ C(t) $ в уравнение для $ x_2(t) $
   			\begin{gather*}
				x_2(t) = \psi_1 + \frac{1}{2}(\psi_2^0 - \psi_1) e^{t} + \tilde{C} e^{-t}.   			
   			\end{gather*}
   			Подставляя краевое условие $ x_2(0) = 0, $ в результате получим
   			\begin{gather*}
				x_2(t) = \psi_1 + \frac{1}{2}(\psi_2^0 - \psi_1) e^t - \psi_1 e^{-t} - \frac{1}{2}(\psi_2^0 - \psi_1) e^{-t} \\
				x_2(t) = \psi_1(1 - e^{-t}) + (\psi_2^0 - \psi_1) \frac{e^t - e^{-t}}{2} \\
				x_2(t) = \psi_1(1 - e^{-t}) + (\psi_2^0 - \psi_1) \sh(t).
   			\end{gather*}
   			И так, теперь, подставив $ x_2(t) $ в систему \eqref{odeSystem}, найдём $ x_1(t). $
   			\begin{gather*}
				\dot{x_1}(t) = \psi_1(1 - e^{-t}) + (\psi_2^0 - \psi_1) \sh(t) \\
				x_1(t) = \psi_1(t + e^{-t}) + (\psi_2^0 - \psi_1) \ch(t) + \tilde{C}.
   			\end{gather*}
			Подставляем краевые условия $ x_1(0) = 0 $ и получаем уравнение для $ x_1(t) $
			\begin{gather*}
				x_1(t) = \psi_1(t + e^{-t}) + (\psi_2^0 - \psi_1) \ch(t) - \psi_2^0.
   			\end{gather*}
   			Для нахождения $ \psi_2^0 $ воспользуемся краевым условием $ x_1(T) = L $
   			\begin{gather*}
				x_1(t) = L \\
				\psi_1(T + e^{-T}) + (\psi_2^0 - \psi_1) \ch(T) - \psi_2^0 = L \\
				\psi_2^0 = \frac{L + \psi_1(sh(T) - T)}{ch(T) - 1}
   			\end{gather*}
   			А для того чтобы найти $ \psi_1, $ воспользуемся краевым условием $ \vert x_2(T) \vert \leqslant \varepsilon. $ Также выше было показано, что $ x_2(t) \geqslant 0, \forall t \geqslant 0. $
   			\begin{gather*}
   				0 \leqslant x_2(t) \leqslant \varepsilon \\
			 	0 \leqslant \psi_1(1 - e^{-T}) + \left(\frac{L + \psi_1(sh(T) - T)}{ch(T) - 1} - \psi_1 \right)\sh(T) \leqslant \varepsilon \\
				0 \leqslant \psi_1 \left[ (1 - e^{-T}) + \frac{sh(T)(sh(T) - T)}{ch(T) - 1} - sh(T) \right] + \frac{L sh(T)}{ch(T) - 1} \leqslant \varepsilon \\
				-\frac{L sh(T)}{ch(T) - 1} \leqslant \psi_1 \left[ (1 - e^{-T}) + \frac{sh(T)(sh(T) - T)}{ch(T) - 1} - sh(T) \right]  \leqslant \varepsilon - \frac{L sh(T)}{ch(T) - 1} \\
				-L sh(T) \leqslant \psi_1 \left[ (1 - e^{-T}) \cdot (\ch(T) - 1) + sh(T)(1 - T - e^{-T}) \right]  \leqslant \varepsilon(\ch(T) - 1) - L sh(T) \\
				-L sh(T) \leqslant \psi_1 g(T) \leqslant\varepsilon(\ch(T) - 1) - L sh(T),
   			\end{gather*}
   			где $ g(T) = (1 - e^{-T}) \cdot (\ch(T) - 1) + sh(T)(1 - T - e^{-T}) = 2(\ch(T) - 1) - T \sh(T). $ Исследуем знак $ g(T), \forall \; T > 0. $
   			\begin{enumerate}
				\item $ g(0) = 0, $
				\item $ g'(T) = \sh(T) - T \ch(T), $
				\item $ g'(0) = 0, $
				\item $ g''(T) = -T \sh(T) < 0. $  			
   			\end{enumerate}
   			Следовательно $ g(T) < 0, \forall \; T > 0 $ и можно разделить неравенство на $ g(T). $
   			\begin{gather*}
   				\frac{\varepsilon(\ch(T) - 1) - L sh(T)}{2(\ch(T) - 1) - T \sh(T)} \leqslant \psi_1 \leqslant \frac{-L \sh(T)}{2(\ch(T) - 1) - T \sh(T)}  			
   			\end{gather*}
   			\item $ \psi_1 = \psi_2^0. $ И так, мы имеем систему
   			\begin{gather*}
   				\begin{cases}
   					\psi_2(t) = \psi_2^ 0, \\
   					x_2(t) = \psi_2^0(1 - e^{-t}), \\
   					x_1(t) = \psi_2^0(t + e^{-t} - 1),
   				\end{cases}  			
   			\end{gather*}
   			с краевыми условиями
   			\begin{gather*}
   				\begin{cases}
   					\vert x_2(t) \vert = \vert \psi_2^0(1 - e^{-t}) \vert \leqslant \epsilon, \\
   					x_1(T) = \psi_2^0(T + e^{-T} - 1) = L,
   				\end{cases}
   				\Leftrightarrow
   				\begin{cases}
   					\left \vert \frac{L(1 - e^{-t})}{T + e^{-T} - 1} \right \vert \leqslant \epsilon, \\
   					\psi_2^0 = \frac{L}{T + e^{-T} - 1}.
   				\end{cases}
   			\end{gather*}
   			Окончательно имеем
   			\begin{gather*}
   				\begin{cases}
   					\psi_1 = \psi_2 = \frac{L}{T + e^{-T} - 1}, \\
   					x_2(t) = \frac{L}{T + e^{-T} - 1}(1 - e^{-t}), \\
   					x_1(t) = \frac{L}{T + e^{-T} - 1}(t + e^{-t} - 1),
   				\end{cases}  			
   			\end{gather*}
   			при 
   			$$
   				0 \leqslant \frac{L(1 - e^{-t})}{T + e^{-T} - 1} \leqslant \varepsilon.
   			$$
   			\item $ \psi_1 > \psi_2^0. $ \\
   			Пусть теперь $ t_{пер} \in (0, T). $  %Имеет смысл рассматривать лишь те $ (\psi_1, psi_2^0), $ для которых $ \exists t_{пер} \in (0, T): \psi_2(t) = 0. $
   			В момент времени $ t = t_{\text{пер}} $ управление $ u $ не определено однозначно. Но заметим, что $ \dot{\psi_2}(t_{\text{пер}}) = -\psi_1 < 0 $ и $ \psi_2(t_{\text{пер}}) = 0. $ Следовательно $ \exists \delta > 0 : \; \psi_2(t) < 0, \; \forall t \in [t_{\text{пер}}, t_{\text{пер}} + \delta]. $ \\
   			Как было показано выше $ \forall t > 0 \; x_2(t) > 0, $ в том числе и для $ t = t_{\text{пер}} > 0. $
   			Следовательно оптимальное управление принимает следующий вид $ u_1^{\star} = 0, u_2^{\star} = k. $ Справедлива следующая система
   			\begin{gather*}
   				\begin{cases}
   					\dot{\psi_2}(t) = -\psi_1 + \psi_2(t)(1 + k), \\
   					\dot{x_2}(t) = - x_2(t)(1 + k),
   				\end{cases}  			
   			\end{gather*}
   			с начальными условиями
   			\begin{gather*}
   				\begin{cases}
   					\psi_2(t_{\text{пер}}) = 0, \\
   					x_2(t_{\text{пер}}) = \psi_1 (1 - e^{-t_{\text{пер}}}) + (\psi_2^0 - \psi_1) \sh(t_{\text{пер}}).
   				\end{cases}  			
   			\end{gather*}
   			Сначала найдём $ \psi_2(t). $ Искать будем методом вариации постоянной. Для начала решаем однородное уравнение
   			\begin{gather*}
   				\dot{\psi_2} = \psi_2(1 + k). \\
   				\frac{d \psi_2}{\psi_2} = (1 + k) dt \\
   				\ln \vert \psi_2 \vert = (1 + k) t + \tilde{C} \\
   				\psi_2 = C e^{(1 + k)t}.
   			\end{gather*}
   			Далее делаем замену $ C \rightarrow C(t). $
   			\begin{gather*}
   				C'(t) e^{(1 + k) t} + (1 + k) C(t) e^{(1 + k) t} = -\psi_1 + C(t) e^{(1 + k) t} (1 + k) \\
   				C'(t) = -\psi_1 e^{-(1 + k) t} \\
   				C(t) = \dfrac{1}{1 + k} \psi_1 e^{-(1 + k) t} + \tilde{C}.
   			\end{gather*}
   			Подставляем $ C(t) $ в выражение для $ \psi_2(t). $
   			\begin{gather*}
   				\psi_2(t) = \dfrac{1}{1 + k} \psi_1 + \tilde{C} e^{(1 + k) t}.
   			\end{gather*}
   			Подставляя начальное уловие, в результате получим
   			\begin{gather*}
   				\psi_2(t) = \dfrac{1}{1 + k} \psi_1 (1 -   e^{(1 + k)(t - t_{\text{пер}})}), \; \forall t >  t_{\text{пер}}.
   			\end{gather*}
   			Теперь найдём $ x_2(t). $
   			\begin{gather*}
  	 			\dot{x_2}(t) = - x_2(t)(1 + k) \\
   				\frac{d x_2}{x_2} = -(1 + k) dt \\
   				\ln \vert x_2 \vert = -(1 + k) t + \tilde{C} \\
   				x_2 = C e^{-(1 + k) t}.
   			\end{gather*}
   			Подставляем краевое условие и получаем, что
   			\begin{gather*}
  	 			x_2(t) = (\psi_1 (1 - e^{-t_{\text{пер}}}) + (\psi_2^0 - \psi_1) \sh(t_{\text{пер}})) e^{-(1 + k) (t - t_{\text{пер}})}, \; \forall t > t_{\text{пер}}. 
   			\end{gather*}
   			
   			Будем искать $ x_1 $ из уравнения 
   			\begin{gather*}
   				\dot{x_1}(t) = x_2(t). \\
   				x_1(t) = -\frac{1}{1 + k} (\psi_1 (1 - e^{-t_{\text{пер}}}) + (\psi_2^0 - \psi_1) \sh(t_{\text{пер}})) e^{-(1 + k) (t - t_{\text{пер}})} + C
   			\end{gather*}
   			Для вычисления константы $ C, $ подставим $ t_{\text{пер}} $ в ранее найденную формулу для $ x_1(t). $ В итоге имеем
   			\begin{gather*}
   				x_1(t) = \frac{1}{1 + k} \left[ (\psi_1 (1 - e^{-t_{\text{пер}}}) + (\psi_2^0 - \psi_1) \sh(t_{\text{пер}})) \right](1 - e^{-(1 + k) (t - t_{\text{пер}})}) + \\
+ \psi_1 (t_{\text{пер}} + e^{-t_{\text{пер}}}) + ( \psi_2^0 - \psi_1) ch(t_{\text{пер}}) - \psi_2^0,
   			\end{gather*}
   			где $ t > t_{\text{пер}}. $ \\
   			Пользуясь тем, что $ \psi_2(t_{\text{пер}}) = 0 $ и формулой для $ \psi_2(t), $  полученной ранее, имеем
   			\begin{gather*}
   				\psi_1 = \frac{\psi_2^0 e^{t_{\text{пер}}}}{e^{t_{\text{пер}}} - 1}.
   			\end{gather*}
   			Подставляем $ \psi_1 $ в выражение для $ x_1(t): $
   			\begin{gather*}
   				x_1(t) = \frac{1}{1 + k} \left[ \psi_2^0 - \frac{\psi_2^0}{e^{t_{\text{пер}}} - 1} \sh(t_{\text{пер}}) \right](1 - e^{-(1 + k) (t - t_{\text{пер}})}) + \\
+ \frac{\psi_2^0 e^{t_{\text{пер}}}}{e^{t_{\text{пер}}} - 1} (t_{\text{пер}} + e^{-t_{\text{пер}}}) - \frac{\psi_2^0}{e^{t_{\text{пер}}} - 1} ch(t_{\text{пер}}) - \psi_2^0 \\ 
				x_1(t) = [\frac{1}{1 + k} \left[ e^{t_{\text{пер}}} - 1 - \sh(t_{\text{пер}}) \right](1 - e^{-(1 + k) (t - t_{\text{пер}})}) + \\
+ e^{t_{\text{пер}}} (t_{\text{пер}} - 1) - ch(t_{\text{пер}}) + 2 ] \cdot \frac{\psi_2^0}{e^{t_{\text{пер}}} - 1}.
			\end{gather*}
			По условию $ x_1(T) = L. $
			Положим
			\begin{gather*}
   				G(t_{\text{пер}}) = \frac{1}{1 + k} \left[ e^{t_{\text{пер}}} - 1 - \sh(t_{\text{пер}}) \right](1 - e^{-(1 + k) (T - t_{\text{пер}})}) + \\
+ e^{t_{\text{пер}}} (t_{\text{пер}} - 1) - ch(t_{\text{пер}}) + 2.
   			\end{gather*} Следовательно
			\begin{gather*}
				\psi_2^0 = \frac{L (e^{t_{\text{пер}}} - 1)}{G(t_{\text{пер}})}.
   			\end{gather*}
   		\end{itemize}
   	\item Пусть $ \psi_2^0 = 0. $
   	При $ \psi_2^0 = 0 $ управление определено неоднозначно. Но можно заметить, что 
   	$$ 
   		\dot{\psi_2}(0) =  - \psi_1.
   	$$
   	\begin{itemize}
   		\item $ \psi_1 < 0. $ \\
   		Из того, что $ \dot{\psi_2} > 0 $ и $ \psi_2(0) = 0, $ следует
		$$
			\psi_2(t) > 0, \; \forall t \in (0, \delta).
		$$
		Из фомулы \eqref{x2(t)} и того, что $ u_1^{\star}  = \psi_2(t), $ следует  $ x_2 (t) > 0,  \; \forall t \in (0, \delta). $
		Из этих двух фактов вытекает 
		$$
			x_2 \psi_2 > 0 \Rightarrow u_2 = 0.
		$$
		Теперь мы знаем оптимальное управление и можем выписать следующую систему
		\begin{gather*}
   				\begin{cases}
   					\dot{\psi_2}(t) = -\psi_1 + \psi_2(t), \\
   					\dot{x_2}(t) = \psi_2(t) - x_2.
   				\end{cases}  			
   		\end{gather*}
   		Как было показано выше, данная система при $ \psi_2^0 = 0 $ имеет следующее решение
   		\begin{gather*}
   				\begin{cases}
   					\psi_2(t) = \psi_1(1 - e^t), \\
   					x_2(t) = \psi_1(1 - \ch(t)), \\
   					x_1(t) = \psi_1(t - \sh(t)).
   				\end{cases}  			
   		\end{gather*}
   		Воспользовавшись краевым условием: $ x_1(T) = L, $ получаем
   		$$
   			\psi_1 = \frac{L}{T - \sh(T)}.
   		$$
   		А из краевого условия $ \vert x_2(T) \vert \leqslant \varepsilon $ получаем ограничение на параметры $ L, T $
   		$$
   			0 \leqslant \frac{L(1 - \ch(T))}{T - \sh(T)} \leqslant \varepsilon.
   		$$
   		\item $ \psi_1 = 0. $
   		В этом случае получаем, что $ \psi_2 \equiv 0. $ Из формулы для $ u_1^{\star} $ и формулы для $ x_2(t) $ $ \Rightarrow x_2(t) \equiv 0. $
   		В данном случае существует только единственное решение: $ L = 0. $
   		\item $ \psi_1 > 0. $ \\
   		Из того, что $ \dot{\psi_2} < 0 $ и $ \psi_2(0) = 0, $ следует
		$$
			\psi_2(t) < 0, \; \forall t \in (0, \delta).
		$$
		Из формулы для $ \psi_2 $ видно, что $ \psi_2(t) < 0, \forall t > 0. $
		Следовательно $ x_2(t) = 0, \forall t > 0. $ В результате имеем, что задача имеет решение при $ L = 0. $		 
   	\end{itemize}
   	\item $ \psi_2^0 < 0. $ \\
   	В силу непрерывности $ \psi_2(t) < 0, \; \forall t  \in [0, \delta]. $ Из формулы для $ u_1^{\star} $ и формулы для $ x_2(t) $ $ \Rightarrow x_2(t) = 0, \forall t \in [0, \delta]. $

	\end{enumerate}
	
	\subsubsection{Случай, когда $ \psi_0 = 0 $ }
	Теперь расмотрим случай, когда $ \psi_0 = 0. $
	Выпешем из условия максимума оптимальное управление $ u^{\star}(t) = (u_1^{\star}(t), u_2^{\star}(t)). $
	\begin{equation*}
		u_1^{\star} = 
		\begin{cases}
			0, & \psi_2 < 0, \\
			[0, +\infty), & \psi_2 = 0, \\ 
			+\infty, & \psi_2 > 0,
   		\end{cases}
   	\end{equation*}
   	\begin{equation*}
   		u_2^{\star} = 
		\begin{cases}
			0, & \psi_2 x_2 > 0, \\
			[0, k], & \psi_2 x_2 = 0, \\ 
			k, & \psi_2 x_2 < 0.
   		\end{cases}
	\end{equation*}	

	Нам интересны только конечные управления $ \Rightarrow \psi_2(t) \leqslant 0, \forall t > 0. $	
	
	\begin{enumerate}
		\item $ \psi_2^0 < 0. $
		\begin{itemize}
			\item $ \psi_1 \geqslant 0 $ и $ \psi_2(t) < 0, \forall t > 0. $ \\
			В таком случае $ \dot{\psi_2}(t) < 0, \forall t > 0. $ Следовательно $ u_1^{\star} = 0 \Rightarrow x_2(t) \equiv 0, \forall t > 0. $ Данная система имеет решение при $ L = 0. $
			\item $ \psi_1 = 0 $ и $ \psi_2(t) = 0, \forall t > 0. $ \\
			В этом случае $ \dot{\psi_2}(t) \equiv 0, \forall t > 0. $
			\item $ \psi_1 < 0. $ \\
			$ \dot{\psi_2}(t) = - \psi_1 \Leftrightarrow \exists t_{\text{пер}}: $ при $ t = t_{\text{пер}}  \; \psi_2(t) = 0. $ \\
			Пусть $ t_{\text{пер}} \in (0, T). $ Тогда   $ \dot{\psi_2}(t_{\text{пер}}) > 0. $ Следовательно для $ \forall \delta > 0 \;  \psi_2(t_{\text{пер}} + \delta) > 0. $ Но тогда $ u_1^{\star}(t_{\text{пер}} + \delta) = +\infty. $ \\
			Пусть $ t_{\text{пер}} \not \in (0, T). $
		\end{itemize}
		\item $ \psi_2^0 = 0. $ Значит $ \dot{\psi_2} = - \psi_1. $
		\begin{itemize}
			\item $ \psi_1 = 0. $ \\
			Возникает противоречие с нетривиальностью вектора сопряжённых переменных.
			\item Рассмотрим случай, когда $ \psi_1 > 0. $ \\ 
			Положим $ t_0 = 0. $ В данном случае $ \dot{\psi_2}(t_0) = -\psi_1 \Rightarrow \dot{\psi_2}(t_0) < 0. $ Следовательно существует окрестность точки $ t_0, $ в которой функция $ \psi_2(t) $ положительна, т.е. $ \exists \; \delta > 0 : \psi_2(t) < 0, \forall t \in (t_0, \delta]. $ Промежуток времени $ [-\delta,  t_0) $ не рассматривается, ввиду ввиду неотрицательности аргумента функции $ \psi_2(t). $ Мы получаем, что $ u_1^{\star}(t) = 0, \;  \forall t \in (t_0, \delta]. $ Следователно $ x_2(t) = 0, \;  \forall t \in (t_0, \delta]. $
			\item Рассмотрим случай, когда $ \psi_1 < 0. $ \\
			Положим $ t_0 = 0. $ В данном случае $ \dot{\psi_2}(t_0) = -\psi_1 \Rightarrow \dot{\psi_2}(t_0) > 0. $ Следовательно существует окрестность точки $ t_0, $ в которой функция $ \psi_2(t) $ положительна, т.е. $ \exists \; \delta > 0 : \psi_2(t) > 0, \forall t \in (t_0, \delta]. $ Промежуток времени $ [-\delta,  t_0) $ не рассматривается, ввиду ввиду неотрицательности аргумента функции $ \psi_2(t). $ Этот случай не представляет интереса, потому что мы рассматриваем только конечное управление, а мы получили, что $ u_1^{\star}(t) = +\infty, \; \forall t \in (t_0, \delta]. $
		\end{itemize}
	\end{enumerate}		
\end{document}
